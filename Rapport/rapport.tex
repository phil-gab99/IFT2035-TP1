\documentclass[11pt, titlepage]{article}
% \usepackage{fullpage}
\usepackage[margin=0.5in]{geometry}
\usepackage{amsmath, amssymb}
\usepackage{microtype}

% Code snippets
\usepackage{listings}
\usepackage{color}

\definecolor{dkgreen}{rgb}{0,0.6,0}
\definecolor{gray}{rgb}{0.5,0.5,0.5}
\definecolor{mauve}{rgb}{0.58,0,0.82}

\lstset{frame=tb,
    language=Haskell,
    showstringspaces=false,
    columns=flexible,
    basicstyle={\small\ttfamily},
    numbers=none,
    numberstyle=\tiny\color{gray},
    keywordstyle=\color{blue},
    commentstyle=\color{dkgreen},
    stringstyle=\color{mauve},
    breaklines=true,
    breakatwhitespace=true,
    tabsize=4
}

\begin{document}

\title{{\Huge \textbf{TP1 - Rapport}}}
\author{Philippe Gabriel \\ Dana Seif Eddine}
\date{30 mai 2021}

\maketitle

\pagenumbering{arabic}
\setcounter{page}{2}

\newpage

\section{Syntaxe de Psil}

La syntaxe de ce langage fonctionnel était assez simple à saisir, surtout
après avoir regardé les exemples simples dans le fichier \texttt{sample.psil}.
Pour la plupart des règles syntaxiques, nous avions presque immédiatement pu
associer une sémantique à chacune de celles-ci. La syntaxe étant préfixe a
grandement simplifié l'analyse d'expressions et leur conversion.

\subsection{Implantation - Conversion de Sexp à Lexp}

Un élément important que nous avions fait avant de commencer notre
implémentation de code était de rapidement survoler la section du code traitant
la première phase déjà fournit, visant à transformer le code source en une
représentation \texttt{Sexp}. Le but était de se familiariser avec les
définitions et fonctions utiles pour compléter la seconde phase. Un premier
défi par la suite était de bien comprendre la représentation \texttt{Sexp}
provenant d'une expression \texttt{Psil}. Pour le surmonter, nous avions
utilisé la fonction \texttt{sexpOf} sur certains exemples de ceux donnés, pour
ensuite en refaire la trace et comprendre l'ordre de construction de
l'expression en \texttt{Psil} vers une \texttt{Sexp}. \\

Par après, la conversion d'une \texttt{Sexp} en \texttt{Lexp} s'est déroulée
assez simplement et rapidement. Après s'être familiariser avec le comportement
de la fonction \texttt{sexp2list}, la conversion des expressions \texttt{if},
\texttt{tuple} et \texttt{fetch} étaient assez directes. \\
Pour l'expression \texttt{call}, afin de satisfaire son caractère curried, une
première solution était de repasser en argument à \texttt{s2l} le premier
argument du constructeur \texttt{Scons}. Pour cela, nous avions modifié le code
comme-ci:

\begin{lstlisting}
s2l (se@(Scons se1 _)) = -- Modification ici
    case sexp2list se
        ...
        (Ssym "call" : e : e1 : []) -> Lcall (s2l e) (s2l e1)
        (Ssym "call" : en) -> Lcall (s2l se1) (s2l (last en))
\end{lstlisting}

Cette implémentation fonctionnait comme souhaité, mais une approche similaire
ne fut pas possible pour \texttt{fun}. De plus, nous avons réalisé que le temps
de calcul pour gérer le cas d'un appel de fonction est loin d'être optimal en
raison du fait que l'on repasse une portion de notre \texttt{Sexp} à
\texttt{s2l} pour la reconvertir avec \texttt{sexp2list} en une liste, ce qui
est du travail inutile puisque l'on possède de la représentation en liste
désirée. Il fut donc nécessaire de légèrement remodifier la définition de la
fonction principale \texttt{s2l} comme suit:
\begin{lstlisting}
s2l (se@(Scons _ _)) =
    let
        selist = sexp2list se -- Modification ici
    in
        ...
        (Ssym "call" : _) -> s2l' se selist
        (Ssym "fun" : _) -> s2l' se selist
\end{lstlisting}

où nous avions simplement dénoté l'expansion de la \texttt{Sexp} sous le nom de
la variable \texttt{selist}. La raison du changement est pour ensuite faire
appel à la fonction \texttt{s2l'}, une fonction auxiliaire capable de traiter
les cas récursifs que l'on retrouve dans \texttt{call} et \texttt{fun}. \\
Finalement, pour l'expression \texttt{let}, il a été important de penser aux
différentes syntaxes de déclarations qu'offrent le langage \texttt{Psil}. Il a
donc été nécessaire de définir une fonction \texttt{s2d} qui traite les
différents cas possibles. \\

En ce qui concerne les types, les cas de \texttt{Bool} et \texttt{Tuple} furent
assez direct. Par contre, pour le type des fonctions, il fut nécessaire, tout
comme dans le cas de \texttt{s2l}, de redéfinir la fonction \texttt{s2t} en
dénotant l'expansion de la \texttt{Sexp} en liste et en définissant une
fonction auxiliaire \texttt{s2t'} pour gérer l'aspect curried du type des
fonctions.

\subsection{Sucre syntaxique}

Après l'implémentation assez direct de la syntaxe du langage, nous nous sommes
tournés vers le sucre syntaxique et les différentes équivalences de syntaxe
qu'offre le langage. \\

Pour ce qui est de la syntaxe des expressions, notre définition initiale
supportait déjà les différentes variantes acceptées. \\

Pour ce qui est de la syntaxe des déclarations, au début de ce projet il ne
semblait pas nécessaire qu'après être rendu à la fin du projet de modifier
notre implantation initiale. Nous avions pris note des différentes équivalences
entre déclarations mais ne savions pas tout à fait pourquoi et comment, à ce
stade, modifier notre définition qui ne faisait que directement traduire une
déclaration en ses \texttt{Lexp} et \texttt{Ltype} correspondants. \\
Lors de l'exécution de tests, il y eut des problèmes dans la vérification et
l'évaluation de certaines expressions. Nous nous sommes souvenus de notre note
initiale, et avions par après modifier les traductions de déclarations de sorte
à les convertir à la forme suivante:

\begin{equation*}
    \begin{aligned}
        & (x \ (x_1 \ \tau_1) \ ... \ (x_n \ \tau_n) \ \tau \ e) && \implies \\
        & (x \ (\tau_1 \ ... \ \tau_n \rightarrow \tau) \ (\text{fun} \
            (x_1 \ ... \ x_n) \ e)) && \implies \\
        & (x \ \tau \ e) && \implies (x \ (\text{hastype} \ e \ \tau))
    \end{aligned}
\end{equation*}

Ce que les implications ci-dessus décrivent:

\begin{itemize}
    \item C'est que la déclaration d'une fonction peut être réécrite en
    spécifiant le type sous une forme currifiée et le corps comme étant une
    fonction;
    \item Qui à son tour peut être réécrite comme toute déclaration de variable
    avec son type;
    \item Qui peut être réécrite l'aide de l'expression \texttt{hastype}.
\end{itemize}
Cette dernière réécriture est celle qui est préférée pour la vérification de
types qui suit et donc \texttt{s2t} ainsi que \texttt{s2t'} ont été redéfinies
afin de réécrire la \texttt{Sexp} en \texttt{Lexp} sous la forme désirée.

\section{Règles de typage}

Les règles de typages fournies étaient assez intuitives dans le choix de
notation, et ont servis de base de la complétion des fonctions \texttt{infer}
et \texttt{check}.

\subsection{Implantation - Vérificateur de types}

Cette partie de l'implantation s'est déroulée très rapidement. En suivant les
règles de typage du langage, il y a cette correspondance directe entre les
règles et l'implantation. Au fur et à mesure que l'on introduisait une nouvelle
expression, des tests rapides à l'aide la fonction \texttt{typeOf} venait
confirmer que notre implantation s'est bien faite.

\section{Règles d'évaluation}

Les règles d'évaluation étaient assez intuitives et les \(\beta\)-réductions
appliquées étaient prises en notes lors de cette phase. La seule règle qui ne
figurait pas dans les instructions était celle propre aux fonctions (pas
l'appel de fonction). Après le survol du code, nous nous sommes basés sur la
forme des fonctions prédéfinies du langage pour déterminer la forme attendue de
l'évaluation d'une fonction. On note également une correspondance directe des
constructeurs de types de \texttt{Value} avec ceux de \texttt{Ltype} ce qui
nous a indiqué précisément le type de résultat de l'évaluation des diverses
expressions.

\subsection{Implantation - Évaluateur}

À la lumière des remarques et notes ci-dessus, l'évaluation des expressions
\texttt{call}, \texttt{let}, \texttt{if}, \texttt{tuple} et \texttt{fetch}
fut assez directe et simple. Le cas qui a prouvé être difficile est celui de
\texttt{fun} pour lequel il n'était pas clair à vue d'oeil comment décrire son
évaluation. Comme première idée, nous avions remarqué que le type semble être
nécessaire pour l'évaluation d'une fonction, et donc avions modifié le code
comme suit:

\begin{lstlisting}
eval2 senv (Lhastype e t) =
    case t of
    Larw _ _ -> eval2fun senv e t
    _ -> eval2 senv e
\end{lstlisting}
Cette idée semblait être la seule solution possible à ce temps, mais on était
pas arrivé à déterminer une forme générale pour les diverses formes de
fonctions. C'est alors que nous avions pensé à une alternative, sans avoir
besoin du type de la fonction, où la valeur de celle-ci sera déterminée au
moment après qu'elle soit définie. Et donc, le code modifié précédemment a été
remis tel qu'il était et notre idée alternative sur l'évaluation d'une fonction
est comme suit:

\begin{lstlisting}
eval2 senv (Lfun a e) =
    \venv -> (Vfun Nothing (\v -> (eval2 (a : senv) e) (v : venv)))
\end{lstlisting}

\section{Tests}

L'étape qui suit à ce point est de tester notre implémentation sur des
exemples. Nous avions employé pour commencer les tests qui figurent dans le
fichier \texttt{sample.psil}. On voyait que tous les tests passaient sauf deux
en particulier:

\begin{itemize}
    \item La vérification de types de l'expression contenant une déclaration de
    fonction faisant référence à deux déclarations qui la précédait (l'avant-
    avant dernier exemple);
    \item La vérification et l'évaluation de l'expression contenant une
    déclaration de fonction faisant référence à une autre déclaration qui la
    suivait (le dernier exemple).
\end{itemize}

Dans les deux cas, il s'agissait de déclaration de fonctions, et le problème
survenait en raison du fait que nous n'avions pas tout à fait bien implémenté
l'équivalence syntaxique de typage, décrite dans la section du sucre
syntaxique. \\

Dans les deux cas, il y avait un problème se posant dans la vérification de
types de déclarations faisant référence à précédentes ou successives
déclarations. Cela a été corrigé après 2 itérations en modifiant légèrement
l'ordre d'instructions et la façon dont on entreprenait la mise-à-jour de
l'environnement de types qui est comme suit:

\begin{lstlisting}
-- Iteration 1 - Erreur dans declarations successives
infer tenv (Llet [] e) = infer tenv e
infer tenv (Llet ((vi, ei) : ds) e) =
    infer ((vi, infer tenv ei) : tenv) (Llet ds e)

-- Iteration 2 - Erreur dans declarations precedentes
infer tenv (Llet ds b) =
    infer ((map (\(v, e) -> (v, infer tenv e)) ds) ++ tenv) b

-- Iteration 3 - Aucune erreur
infer tenv (Llet ds b) =
    let
        (tenvn, tenvt) = unzip tenv
        (vars, exps) = unzip ds
        tenvn' = vars ++ tenvn
        tenvt' = (map (\e -> infer (zip tenvn' tenvt') e) exps) ++ tenvt
    in
        infer (zip tenvn' tenvt') b
\end{lstlisting}
Les cas d'erreurs qui sont survenues s'apparente au \texttt{let} et
\texttt{let*} du langage \texttt{Lisp}. Avec la dernière itération, l'idée d'un
\texttt{letrec} a été implémentée. On remarque, si l'on suit les instructions
de près, comment la procédure diffère d'une itération à l'autre. \\

Dans le second cas, un problème d'évaluation survenait dans le cas où une
déclaration faisait référence à une déclaration successive. La solution suit de
très près celle discutée pour le cas de vérification de types:

\begin{lstlisting}
-- Iteration 1 - Erreur dans declarations successives
eval2 senv (Llet [] b) = (eval2 senv b)
eval2 senv (Llet ((v, e) : ds) b) =
    \venv -> ((eval2 (v : senv) (Llet ds b)) (((eval2 senv e) venv) : venv))

-- Iteration 2 - Aucune erreur
eval2 senv (Llet ds b) = \venv ->
    let
        (vars, exps) = unzip ds
        senv' = vars ++ senv
        venv' = (map evlt exps) ++ venv
        evlt = \v -> ((eval2 senv' v) venv')
    in
        ((eval2 senv' b) venv')
\end{lstlisting}
Suite à ces corrections, tous les tests du fichier ont passé. \\

Cependant, nous nous étions pas encore penchés sur les détails de la syntaxe du
langage, surtout dans le cas d'expressions contenant des erreurs. Par exemple,
notre définition initiale de la conversion d'un type \texttt{Sexp} en un
\texttt{Ltype} était trop vague et laisser donc passer des syntaxes incorrects:

\begin{lstlisting}
-- Version 1 - Definition trop vague
s2t' :: Sexp -> [Sexp] -> Ltype
s2t' se selist =
    case selist of
        (ta : Ssym "->" : tr : []) -> Larw (s2t ta) (s2t tr)
        (ta : Ssym "->" : tr) -> Larw (s2t ta) (s2t' se tr) -- Erreur ici
        _ | (last (init selist)) == Ssym "->" ->
              Larw (s2t (head selist)) (s2t' se (tail selist))
          | otherwise -> error ("Unrecognized Psil type: " ++ (showSexp se))

-- Version 2 - Definition rigoureuse
s2t' :: Sexp -> [Sexp] -> Ltype
s2t' se selist =
    case selist of
        (ta : Ssym "->" : tr : []) -> Larw (s2t ta) (s2t tr)
        _ | (last (init selist)) == Ssym "->" ->
              Larw (s2t (head selist)) (s2t' se (tail selist))
          | otherwise -> error ("Unrecognized Psil type: " ++ (showSexp se))
\end{lstlisting}
où l'on remarque ici que la seconde ligne dans le corps du \texttt{case} peut
laisser passer des syntaxes incorrectes. Par après, la révision de
l'implantation des divers \texttt{Sexp} et comment elles sont convertis a été
revisités et plusieurs cas similaires ont été corrigés

\section{Détails sur différents choix d'implémentation}

Suite à la précédente révision qui a mené à gérer différents cas de syntaxe,
certains questionnements sur la syntaxe qui ne sont pas spécifiés dans la
donnée du travail persistent. On tente ici d'expliquer certains de ces
questionnements ainsi que nos décisions par rapport à ceux-ci. \\

Un premier questionnement fut celui du parenthèsage d'expressions. Pour ce qui
est des expressions ou types atomiques (un nombre, une variable, un type
primitif, ...) on se questionnait sur la validité des jugements suivants:

\begin{align}
    \text{Int} & \stackrel{?}{\equiv} \text{(Int)} \tag{Type nombres entiers} \\
    \text{Bool} & \stackrel{?}{\equiv} \text{(Bool)} \tag{Type booléens} \\
    n & \stackrel{?}{\equiv} (n) \tag{Un entier} \\
    x & \stackrel{?}{\equiv} (x) \tag{Une variable}
\end{align}
Après réflexion, il nous a semblé être plus approprié d'empêcher le
parenthèsage de la sorte pour les expressions atomiques puisque les parenthèses
dans notre langage sont significatives \(\implies\) elles indiquent
l'application du constructeur de types \texttt{Scons} composés de deux 
\texttt{Sexp}. Ces exemples abordés étant atomiques impliquent qu'il ne s'agit 
que d'un  \texttt{Sexp} (\texttt{Ssym} ou \texttt{Snum} selon le cas). \\

Un second questionnement est celui traitant sur l'expression \texttt{let} du
langage. Le langage fonctionnel Haskell permet une syntaxe intéressante:
\begin{lstlisting}
let in body
\end{lstlisting}
On remarque qu'aucune déclaration n'a été faite, ce qui rend l'expression
\texttt{let} inutile. Est-il pertinent que notre langage adopte un tel 
comportement? Après avoir noté la structure du constructeur de types
\texttt{Llet} pour une \texttt{Lexp}, on voit que les arguments s'agissent
d'une liste de déclarations et du corps à exécuter. Rien n'empêche que la liste
soit vide, car elle demeurera correctement typée. Cela nous a donc encouragé à
également adopter cette approche. \\

Un troisième questionnement aborde le cas de construction de tuples. Par un
raisonnement similaire à précédemment, on remarque que Haskell permet l'usage
d'une construction particulière de tuples:
\begin{lstlisting}[language=bash]
ghci> :t ()
() :: ()
ghci>
\end{lstlisting}
Le tuple vide est une construction valide du langage. Et comme dans le cas
précédent, on remarque que \texttt{Ltuple} prend une liste de \texttt{Lexp} en
paramètre, ce qui n'empêche pas celle-ci d'être vide. Et bien sûr, comme
démontré ci-haut, le type d'un tuple vide est tout simplement \texttt{Ltup []}.
Nous avions donc décider d'adopter cette mesure également. Cette donnée semble,
à première vue, inutile pour notre langage. Cela en raison du fait qu'il n'est
même pas possible de la manipuler avec l'expression \texttt{fetch} puisqu'il
n'y a rien à assigner. La seule utilité est vraiment peut-être de l'assigner à
une autre variable ou la retourner comme résultat. Si l'on pensait à étendre
le langage Psil, peut-être qu'elle s'avèrerait plus utile.

\section{Notes sur la modification du code}

Il n'y a pas eu de grandes modifications du code source déjà fournit. Les
exceptions ont été expliquées dans les précédentes sections du document. Sinon,
une autre légère modification amenée est simplement le remplacement de certains
des messages d'erreurs sur la syntaxe d'expressions par des fonctions pour
générer le message correspondant de sorte à alléger un peu la lecture du code.

\end{document}